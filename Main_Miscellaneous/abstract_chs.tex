
% !TeX root = ../main.tex
\begin{chineseabstract}
\footnotetext{*本研究得到某某基金(编号:)的资助}
% \astfootnote{本研究得到某某基金(编号:)的资助}

随着毫米波技术的发展和人们对日常生活各方面需求的进一步提高,基于深度神经网络的毫米波感知在动作识别等方向得到了广泛运用,而域适应可以利用已有含标签的数据克服不同反射环境对深度网络感知效果的负面影响。而当下对于个人隐私的高度重视也对该技术提出了新的挑战。因此,无源域适应的毫米波动作识别技术具有非常重要的实用价值。

本文研究了基于无源域适应的毫米波动作识别技术,用于智慧家庭设备、医院患者检测、商场购物结算等场景下的动作感知任务。该方法利用深度学习方法,在只获取含标签数据训练后的网络模型的条件下,通过学习训练集数据表征对模型进行微调,从而在提升模型不同反射的环境中对动作的识别精度。本文的主要研究内容有两个方面:

1)提出了面向多环境数据辅助单目标环境动作识别场景的基于联邦学习方式的域适应方法。首先在本地分别用不同环境下含标签数据训练模型,然后将各模型参数上传至云端对目标环境数据进行分类预测,通过投票机制确定本地各环境模型的贡献度,从而进行参数聚合得到符合目标环境的动作识别模型。该方法可以更好地利用到多环境的优势进行远端联合建模,且保证了数据的隐私性。

本文在不同环境下自行采集的毫米波数据中开展了大量实验,并与该领域相关的工作进行了比较分析。结果表明,本文所提出的方法在不使用已知环境数据的条件下,能够利用调整模型参数的方式直接完成域适应目标,提高目标环境下的动作识别精度,并且在评估指标方面好于基准方法,证明了方法的有效性。

本文做出了以下贡献:

\chinesekeywordstype{无线感知;动作识别;深度学习;无源域适应}{应用探究}

\end{chineseabstract}

