% !TeX root = ../main.tex
\xiaosi

\xchapter{绪论}{Introductions}

\xsection{研究背景及意义}{Backgrounds}

物联网(IoT)技术的发展标志着智能设备和系统互联的新时代。通过无线感知技术,如RFID、毫米波、蓝牙和Wi-Fi,
物联网设备能够感知环境变化、收集数据并进行通信,实现自动化管理和智能化操作。随着5G网络的推出和边缘计算、
人工智能技术的进步,物联网的数据处理和分析能力得到极大增强,推动了其在众多领域的广泛应用。物联网和无线感知技
术的结合不仅提高了生活和工作的便利性,还为解决复杂问题和推动创新开
辟了新途径。未来,随着技术的不断进步和应用标准化的推动,物联网预计将实现更深入的集成和广泛的应用,为社会经济
发展带来深远影响。

无线感知技术作为物联网技术的关键感知手段,随着无线通信技术的快速发展和智能设备的广泛应用,也取得了显著进展,
基于毫米波的动作识别技术\cite{zhang2023survey}就是其中一个重要的分支:毫米波雷达发射和接收电磁波,捕获目标对象的动作信息。雷达信号在
遇到对象时反射,通过分析这些反射信号,可以提取出动作特征。然后,利用信号处理技术,将这些特征转化为可分析的形式。
最后,采用机器学习或深度学习算法对特征进行分类或识别,实现对特定动作的准确识别。这项技术具有许多优势:
与传统的接触式传感器监测和识别技术相比,基于毫米波的动作识别技术不需要与人体直接接触,减少了传染病的传播风险,
也提高了被感知用户的便捷性和舒适性。同时,毫米波雷达能够穿透烟雾、尘埃等环境,即使在视线受阻或光线不足的情况下
也能准确识别动作,这使得它在复杂或恶劣环境下具有出色的适用性。毫米波雷达对于对象速度和距离的精准测量,也保证了
动作识别的高度准确性。得益于上述方面的显著优点,毫米波动
作识别领域产生了许多结合实际情景的优秀工作\cite{li2019bi,an2021mars,liu2020real},并展现出巨大的应用潜力和
社会价值:

健康监测和医疗保健:在医疗保健领域,基于毫米波的动作识别技术可以用于患者的跌倒监测,以及康复训练的辅助等,帮助医生远程监测患者的康复进度,这对于提高医疗服务质量和效率都具有重要意义。

智能家居与办公自动化:在智能家居系统中,基于毫米波的动作识别可以实现远程控制家电、灯光和安全系统等,提高生活便利性和舒适度。在办公自动化领域,该技术可以用于会议室的自动管理、能源的高效使用等。

公共场所非接触人机交互:在商场等场所中,基于毫米波的动作识别可以实现无接触操作,在流行性疾病爆发时,可以有效减少因触碰公共设施而感染病毒的风险。在新冠疫情爆发后,这些处于卫生健康方面的举措更加值得被注意与参考。

尽管基于毫米波的动作识别技术具有众多优势,但它也存在一些局限性和挑战,其中环境问题最为关键。虽然毫米波雷达能够穿透某些物质,但它仍然可能受到环境
因素的干扰,如多径效应(信号反射造成的干扰)、湿度和温度变化等,这些因素可能会影响系统的性能和准确性。

深度学习领域中的域适应技术被认为是一种克服模型对于特征分布存在差异的数据预测结果失准问题的方法,对应在无线感知的问题设置下,已知环境对象的数据被称为源域,需要识别的环境对象数据被称为目标域。该技术通常涉及到一些手段,让模型能够学习到数据中更为普遍、本质的特征。域适应最初主要用于计算机视觉领域,目前也被证明在毫米波动作识别领域具有作用,近年来许多相关工作都被提出。但这些工作大都需要获取一定源环境含标签的数据参与模型训练。尽管毫米波动作识别技术不捕捉人体面部或身体细节,从而在一定程度上保护了用户的个人隐私,但它仍然能够收集关于个人行为和习惯的敏感信息。例如,个人的行为模式、日常活动习惯甚至健康状况都可能通过动作识别技术间接获得,引发用户对于隐私问题的担忧。同时,如果这些敏感数据未经加密或在没有足够安全措施的情况下被传输和存储,也存在着被黑客攻击窃取和数据泄露的风险,可能导致数据遭到未经授权的访问或滥用。如果缺少源环境数据,将缺乏对不同环境之间数据差异的直观度量,导致传统的域适应方法的性能大幅度下降。

为了解决上述问题,本文提出了一种基于无源数据域适应的毫米波动作识别方法。该方法将域适应的侧重点放在对目标环境数据自身所包含特征信息的无监督学习上。针对动作识别的实际场景对域适应方法实现的要求,分布设置不同的处理策略,在不直接利用已有环境数据的条件下,利用联邦学习、伪标签等方法调整目标环境模型参数,改善模型在目标识别环境上的性能。因为目前缺乏公开的毫米波人体动作数据集,同时出于对实际应用意义的考虑,本文在自主实验收集的多个环境下若干名志愿者的毫米波动作数据集上进行了实验,验证了所提出方法的有效性。同时本文与相关领域其他方法进行了对比实验,体现出本方法的优势。

\xsection{国内外研究现状}{Hello}

本文涉及的无源数据域适应技术主要涉及深度学习的内容。随着深度学习技术的发展,计算机视觉领域各种问题场景下的相关技术方法都已非常成熟。目前被提出的许多涉及跨场景无线感知方向的工作也都受到这些方法的影响,但这些工作中缺乏了缺乏源域数据的情况下对多环境跨域动作识别任务的研究。下面将从分别从计算机视觉域适应和无线跨场景感知两个领域方向陈述国内外相关研究现状。

\xsubsection{基于视觉的域适应}{Hello}

作为一项最初主要用于解决视觉领域的技术,域适应目前更多采用深度学习作为解决不同领域之间差异问题的方法。
然而现实应用场景对模型的训练数据提出更严格的要求:许多情况下不同领域的数据并不满足独立同分布的条件,同
时视觉领域一些场景的数据集往往很难获得标注导致了许多数据缺乏对应标签,这些问题超出了传统方法的
假设范围\cite{liu2022deep},对预测性能提出了挑战。为了应对这种情况,近年来许多相关研究人员都以
无监督形式为主要方向。在这种问题设置中,深度模型利用一个领域的数据集进行训练,但需要在所有标签未知的
领域数据集上进行预测,通过特征表示的转换、对抗性训练等技术来找到一种对领域偏移不敏感的表征方法,从而提
升模型在未知数据上的泛化能力。Xu等人\cite{xu2020adversarial}利用了深度学习领域的混合(Mixup)方法,将源域与目标域数据进行线性
插值实现数据增强,提高模型在目标域的泛化能力。Zhu等人\cite{zhu2017unpaired}在生成对抗网络的基础上,设
计循环对抗神经网络(CycleGAN)通过额外添加逆向映射关系,试图在保留图片内容结构的基础上学习新的图片风格,过程
中引入了循环一致性约束确保对两种映射的训练是一致的。Yang等人\cite{yang2023tvt}首次尝试使用Vision Transformer解决无监督域适应任务,
并加入可迁移学习模块同时捕获迁移性特征和判别性特征,提升模型迁移能力。

许多无监督域适应工作也从半监督学习中获得指导,着眼于对特征空间对齐以外的探索,例如对伪标签、熵最小化等方法的应用。
伪标签即使用当前模型为目标域不可知的数据预测,经置信度判别后生成伪标签,从而与源域数据共同参与模型训练。Han等人\cite{han2019deep}利用
余弦相似度作为指标为目标域样本特征计算类聚类中心,之后计算各样本与中心的余弦相似度,将最小值所对应中心类别作为该
样本的伪标签。可以看出,该类方法的精度对伪标签本身的准确性很高,不准确的伪标签可能会给模型学习带来灾难性的误差。熵
最小化利用损失约束模型的预测更自信,即熵值更低。Jin等人\cite{jin2020minimum}提出了一种域适应任务通用的最小类混淆损失函数(MCC)以训练模
型分类器混淆对目标域上正确类和模糊类之间的预测的趋势,关注模型预测类别的确定性。这类方法存在过度信赖自信但不准确的
模型而影响性能的风险。同时,和传统设置的无监督方法一样,这些方法缺乏对于没有源域数据场景跨域任务的处理手段。

随着视觉领域的发展,域适应技术在处理不同特定问题的过程中也出现了新的趋势。现代技术发展对于隐私保护的重视程度一直在
提高,尤其是医疗等严格遵守隐私条例、数据传输约束或数据专有化的场景,标记的数据通常受到限制。这需要考虑到在无法获得
源域数据的前提下探索将不同领域数据表征进行对齐的可能性。Liang等人\cite{liang2020we}利用了半监督学习的算法,提出在预训练完成后固定模型
分类器,通过信息最大化损失函数微调特征提取器、对数据标记伪标签的方式提升模型泛化性能。Xia等人\cite{xia2021adaptive}提出设置多分类器区分数
据样本所属领域,利用自适应对抗推理和对比类别匹配的方式在多分类器和特征生成器之间建立对抗关系,使模型学习到符合多分类
器边界的域无关特征。Ding等人\cite{ding2022source}将源域模型分类器的权重向量作为锚点,在源域数据符合类高斯分布的假设下,利用锚点与目标域数
据的均值与协方差模拟构建源域特征分布,采用对比域差异的损失函数(CDD)显式地将不同域的数据在特征空间对齐。然而,在实
际动作识别的场景下,会需要同时涉及多个环境的数据同时进行训练,许多无源域适应算法无法应对这种情况的挑战。

\xsubsection{基于无线的跨域感知}{Hello}

环境对无线信号有显著影响,包括信号强度的衰减、信号路径的多样化以及外部干扰的介入。环境中的障碍物如墙面和金
属物体等能够吸收和反射无线信号,导致信号弱化。移动中的物体,如人,会改变信号的传播方向,造成多径效应,对接收
信号的稳定性造成影响。同时,其他电磁源的存在可能引起信号干扰,进而降低通信效率。因此,在设计无线通信系统时,必
须考虑到这些环境因素,以保证通信的可靠性和效率。但在实际情况下,无法保证每个感知场景都符合信号传播的理想条件,这
导致了采集到的无线信号包含了大量环境的噪声。因此,在不同的环境下会获得具有不同特征的信号数据。同时,无限感知领域
缺乏视觉领域中具有一定规模的跨域数据集\cite{venkateswara2017deep,sakaridis2019guided,sakaridis2021acdc},考虑到现实中无线信号采集带来的较大时间开销,在所有环境都进行大量
数据收取无疑是不合理的。研究人员针对无线感知的跨场景问题也做了很多探究,目前主要通过信号处理和域适应的方法处理
该问题。

\subsubsection{基于信号处理的跨场景感知}

该类方法主要通过对获取信号中的信道状态信息(CSI)预处理以提取数据中与环境无关的特征进行学习,从而克服无线感知过程
中存在的环境依赖问题。Tan等人\cite{tan2016wifinger}利用时频域变换、离散小波变换的方式消除了CSI中部分与环境有关的噪声,同时通过找寻同一
分类数据最佳对齐作为数据的主要特征以识别手势动作。Yang等人\cite{yang2018fine}通过类估计基空间奇异值分解(CSVD)的方法对CSI进行重构
消除部分环境信息,再采用非负矩阵分解(NMF)获取各类数据的特征分布以进行活动识别任务。Li等人\cite{li2018wifit}从多普勒频移和CSI中包
含的幅度、相位差信息中获取了丰富特征共同作为输入从而对活动进行分类预测。基于信号处理的工作给之后的工作提供了许多数
据层面上环境因素消除的方法参考。然而它们对环境依赖的消除是有限的,还需要从学习特征的方面进行研究以提高整体精度,同
时部分方法也存在设备复杂、难以部署的问题。

\subsubsection{基于域适应的跨场景感知}

该类方法从深度学习中的域适应技术着手,训练模型学习经预处理的无线信号数据中包含的域无关特征。Jiang等人\cite{jiang2018towards}基于对抗的思路
设计了环境无关框架EI,由特征提取器和两个分类器组成,其中分类器分别用于判别样本对应的人类活动种类和环境领域。Ding等人\cite{ding2020rf}提出
了由双路径基础网络和基于度量的元学习框架组成的RF-Net。前者将空间模块与基于注意力的时间模块相结合,用于从时域和频域
学习数据包含的时空特征,后者则通过特定的射频参数模块训练准确的距离度量,从而提升模型在新环境下的泛化能力。Bhalla等人\cite{bhalla2021imu2doppler}利用了较为广泛的智能
手表IMU公开数据集,通过多模态数据的高维特征对齐完成了毫米波雷达传感器的活动识别系统IMU2Doppler。Zhou等人\cite{zhou2022target}受到半监
督学习的启发,提出了基于不确定性的内集划分方法,通过与元学习相结合来更好利用被标记的样本,同时引入一种动态伪标签
策略,将目标域未标记样本纳入训练。然而目前的方法大都是在源域数据可用的基础上进行设计的,这无法满足许多场景对于数据
隐私保护的实际需要。采用多模态迁移的IMU2Doppler则更多展示了该类方法的应用可能性,它的预测精度与其他的无线感知工作
相比有一定的差距。

\xsection{论文研究内容}{Hello}

本文主要研究内容为基于无源域适应的毫米波动作识别方法研究,在现有的域适应和无线感知方法的基础上,结合毫米波动作识别
在实际应用中存在的隐私问题和多方合作问题进行了研究。

本文的研究内容主要分为以下两点:

1)提出了面向多环境数据辅助单目标环境动作识别场景的基于联邦学习方式的域适应方法。首
先在本地分别用不同环境下含标签数据训练模型,然后将各模型参数上传至云端对目标环境数据进行分
类预测,通过投票机制确定本地各环境模型的贡献度,从而进行参数聚合得到符合目标环境的动作识别模型。同时考
虑到多源域情况下各个环境数据来源丰富,在进行联邦学习的过程时也需要注重对模型预测的普适性的提升,
因此本方法又设计了一种源域域适应模块在一定程度上提升源域模型对自身
数据的预测精度。该方法可以更好地利用到多环境的优势进行远端联合建模,且保证了数据的隐私性。

2)

以上两个研究内容在自行采集的毫米波数据集上进行了大量的实验,验证所提出的毫米波动作识别方法的有效性,并于其他相关技术进行了对比分析,证明了本文所提出方法的优势。

\xsection{论文组织结构}{Hello}

本文将分五个章节对研究内容进行介绍,具体如下:

第一章为绪论部分。首先确定基于无源域适应的毫米波动作识别方法研究的研究背景与意义,其次对国内外视觉领域域适应技术和
无线领域跨域感知技术的发展现状进行了总结概括,分析了当前方案的特点和存在的局限,进一步说明了本文的研究意义。最后。对本文
的本文的主要研究内容进行了简要描述。

